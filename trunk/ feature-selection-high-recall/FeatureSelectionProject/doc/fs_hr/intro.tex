\section{Introduction}

Feature selection is important in many practical supervised learning
settings to address computational constraints (e.g., in large-scale or
online learning) or to prevent overfitting when the data may be
high-dimensional but contain relatively few samples (e.g., as may
occur in medical or bioinformatics domains where features are abundant
but data is costly to obtain)~\cite{guyon_jmlr03}.

While a wide variety of feature selection methods have been proposed
in the literature, little research seems to focus on feature selection
specifically targeted to improve recall (minimization of false
negatives) in binary classification.  One reason for this is simply
that most feature selection methods are agnostic to the particular
supervised learning task -- applying to tasks from classification to
regression -- and hence are not focused on performance properties
specific to binary classifiers.  Yet, recall is an important aspect of
many binary classification problems (e.g., minimizing false negatives
in the identification of cancerous tumors) and it is critical to have
feature selection algorithms that can target features which provide
high recall coverage of the data.

Naturally though, it would not make sense for a classifier or feature
selection method to focus on recall alone since the classifier which
always predicts \emph{true} (independent of the data) obtains optimal
recall performance.  To trade off precision with recall in order to
maintain high accuracy, one often uses a geometric average of the two
instead (known as F-score) but in practice most classifiers directly
optimize accuracy or some surrogate, e.g., a convex surrogate of 0-1
loss as in the SVM or maximum (conditional) likelihood as in Naive
Bayes or logistic regression.  So how can one achieve a high recall
SVM, Naive Bayes, or logistic regression classifier that already has a
well-defined accuracy-focused optimization criterion?  The idea we
pursue in this paper is that feature selection can help modulate the
recall performance of existing classifiers by encouraging selection of
features that cover more of the true cases in the data (hence
discouraging false negatives harmful to recall).

In this paper we propose a novel probabilistic voting model of feature
selection with the intent of encouraging false negative reduction
while still focusing on accuracy.  We argue that choosing features so
as to maximize likelihood objectives w.r.t. this model provides an
effective method for high recall feature selection.  Specifically, for
each objective, we derive an efficient formula for feature selection
in the filtering framework (i.e., greedy forward selection) and we
empirically compare the resulting feature selection criteria to a wide
variety of existing filtering methods.

Our results demonstrate that our high recall feature selection
approach does indeed improve recall with improvements most significant when 
there is a high feature to data ratio.  Such results provide a novel
and efficient feature selection algorithm to target high-recall
classification tasks.
% TODO: elaborate more

\COMMENT

% Consider the feature set jointly, weighted voting model

% We get a heavier weighting of effect of data... a feature that pushes more things to their
% class improves recall... so summation is reason for recall.  Each data gets it's own feature vote.

% We get best performance with prob. classifiers, due to alignment of assumptions

% We get best results with conjunction on high ratio of features to data
% - conjunction forces agreement and reduces noise
% - summation encourages coverage of the data
% ... this works best when lot's of features, let data vote, but get probabilistic agreement

% Show anecdotal results for Newsgroup and Spect logistic... we're best across both



%can switch in columns, but typically predicting more false hurts recall and more true hurts precision

%So precision and recall become secondary objectives after accuracy... provides flexibility in feature selection that has not been explicitly considered.

% Yet on other hand certain methods must have done better/wose b/c of implicit assumptions that we should analyze later.

key insight: literature suggests redundancy is bad idea, but in fact, it is critical for precision (implies agreement)

how to strongly justify the model and objectives we use?  voting?

Good discussion: http://jmlr.org/papers/volume13/brown12a/brown12a.pdf

- Feature selection motivations
** too little data (e.g., bioinformatics, or text with few labels)
** features costly to evaluate
** explainability
** minimize feature elicitation for future data collection (questionnaires, medical tests -- blood tests)
** computation / Big Data
** mobile devices (RAM, computation)
** linear speed (LibLinear 1000x faster than LibSVM)
** poly kernels without kernelized approach without loss of evaluation efficiency
** need to argue SVM/logistic L1 not always best... not always logistic regression/SVM problem, also eval efficient, but training not, so 
inefficient for large-scale or resource-constrained training... show empirical better than L1, also more control over amount of feature 
selection (need a tool that implements full regularization path -- constrains optimization method)


Filtering-based feature selection methods ...


"A Probabilistic Model for High Precision and Recall Feature Selection"
"A Probabilistic Model/Perspective of Feature Selection for Trading off Precision and Recall"
Conventional feature selection methods choose informative features or feature sets and empirically lead to good accuracy improvements, but 
few focus directly on precision or recall (or mention that they're method is actually biased towards one vs. other).  In this paper we 
propose a new probabilistic classification model based on a logical or weighted voting approach that directly encourages feature selection 
according to the continuum between pure precision and pure recall.  Supports filter, wrapper, and embedded approaches (where we create a 
new high precision/recall variant of Naive Bayes).  Suggests a way of analyzing existing approaches and identifying what they are 
optimizing.  In fact, MI (or other independent greedy methods) appear to be maximizing precision while MRMR appears to maximizing recall.
- Idea: we have a generative model... write down how we want features to generate the class... now the only leverage we have is to select 
features to maximize expected (log) likelihood
** if we use an aggregation function that allows any feature to falsify (noisy-and), we encourage agreement (precision)
** if we use an aggregation function that allows any feature to tip the vote to true (noisy-or), we encourage coverage (recall)... noisy-
max not multiclass unless we're ordinal (I think)
** we can trade off precision and recall by requiring n-out-of-k features, n=k/2 is majority vote... does this correspond to F-score or 
accuracy (can we compare beta & k, 3d plot showing F_beta for k?)
** of course probabilistic since class predictions are latent so we get a soft version (noisy-or, noisy-and) ... clearer derivation of 
noisy-or, noisy-max etc. than direct specification?  I think so.
** maximizing overall likelihood so will try to get features that correctly classify more of the data (especially where votes of others are 
weak)
** can do a probabilistic analysis of the 4x4 contigency table
** a new naive classifier, how do we do overall as a naive classifier?  accuracy will tell us this, can tune k though for whatever metric 
we care about!  interesting as a classifier since we look at total probability mass of at least k votes... what does this buy us over 
standard naive Bayes?  Should evaluate.
==> is there a way to do DP for dynamic programming for Bayesian model averaging over all class inclusions/exclusions?
... for each feature can leave in or leave out -- if leave out get 1, otherwise probability (sum over all models, probability of prediction 
* normalized likelihood of that model vs. all others -- add in feature one at a time?)
... would this also work for query expansion?  some sort of DP MAP query for jointly most useful retrieval set?
** online learning... discrete parameter
** spectrum/contiuum of tradeoffs for feature selection
- To explore: correlated features, diversity of features (MRMR) -- good for recall
- Filter method
- Wrapper method
- Embedded method - use a naive independence classifier
  (do we unify view for embedded method?  we could maximize likelihood for voting classifier	
- Computational tricks worth noting (extreme cases and 1- trick for probabilities, DP)
- In general we want to optimize precision/recall tradeoff (extreme solutions are trivial), n-out-of-k interpolates F-score... really?  Can 
we show a strong mathematical connection?
- Intepretation as a voting classifier... then majority vote is k/2-call@k
** precision, recall, what is f-score?  pure precision/recall not ideal
** reinterpretation of existing methods as focusing more on precision (MI) or recall (MRMR)
** does trade-off between precision and recall yield an F-Beta score
** should probably write out a direct classification approach with y being I[y = y_1 v ... v y_k] and then maximize E[p(y*|...)] or 
likelihood.  This model supports all view in the proposed paper.  Which is better?
** will get the effect that optimization only matters when new feature can swing the vote... for recall this will be false negatives, for 
recall this will be false positives, for majority vote, it will be borderline cases.
** recall encourages diversity... explains why diverse feature selection methods don't always improve accuracy.  If we wanted precision, we 
don't mind redundancy, in fact we encourage it.
** start with and, or, generalize to weighted voting scheme
** an information-theoretic intepretation?  expectation over data of log probability should yield this
** what sort of similarity metrics do we get?
** need to motivate directly that the logical/voting function encourages precision vs. recall... that with disjunction, new features will 
only address correcting false negatives and likewise for precision and conjunction and correcting false positives
** what sort of features do we select... analyze 2 newsgroups
***
- A new naive independence voting-based classifier with tunable objective and objective-oriented feature selection via a voting scheme
- A new naive independence voting-based classifier for the precision/recall trade-off
** feature selection for cases other than accuracy!
** efficient, supports different objectives from precision to recall and in-between (F-score?)... is the objective just for feature 
selection?
** directly supports feature selection... important in independent classifiers
** supports online computation... and online feature selection scoring
** supports discrete and Gaussian models
*** we could also modify the model to make the logical function (weighted voting explicit)... can marginalize this out... in this way the 
prediction matches feature selection
** learning parameters -- for directed model, as usual
** how to work out argmax_y* P(y*|...) -- looks like its just going to multiply probabilities for each class -- is this another view of 
Naive bayes
** how to work out feature selection -- y* observed...
*** could choose fb to maximize logical function in expectation... expected correctness?
*** could choose fb to maximize likelihood -- would be log version of above with logical function for hidden variable
*** after doing extreme cases, allow general voting, not just all-or-none

*** training recall/precision directly a bad idea... hurts accuracy, but can select features so as to focus on one of these
*** we focus on the filter method in this work
\ENDCOMMENT
