%%%%%%%%%%%%%%%%%%%%%%%%%%%%%%%%%%%%%%%%%%%%%%%%%%%%%%%%%%%%%%%%%%
%%%%%%%% ICML 2014 EXAMPLE LATEX SUBMISSION FILE %%%%%%%%%%%%%%%%%
%%%%%%%%%%%%%%%%%%%%%%%%%%%%%%%%%%%%%%%%%%%%%%%%%%%%%%%%%%%%%%%%%%

% Use the following line _only_ if you're still using LaTeX 2.09.
%\documentstyle[icml2014,epsf,natbib]{article}
% If you rely on Latex2e packages, like most moden people use this:
\documentclass{article}

% For figures
\usepackage{graphicx} % more modern
%\usepackage{epsfig} % less modern
\usepackage{subfigure} 

% For citations
\usepackage{natbib}

% For algorithms
\usepackage{algorithm}
\usepackage{algorithmic}
%\usepackage[noend]{algpseudocode}

% As of 2011, we use the hyperref package to produce hyperlinks in the
% resulting PDF.  If this breaks your system, please commend out the
% following usepackage line and replace \usepackage{icml2014} with
% \usepackage[nohyperref]{icml2014} above.
\usepackage{hyperref}

% Packages hyperref and algorithmic misbehave sometimes.  We can fix
% this with the following command.
\newcommand{\theHalgorithm}{\arabic{algorithm}}

% Employ the following version of the ``usepackage'' statement for
% submitting the draft version of the paper for review.  This will set
% the note in the first column to ``Under review.  Do not distribute.''
\usepackage{icml2014} 
% Employ this version of the ``usepackage'' statement after the paper has
% been accepted, when creating the final version.  This will set the
% note in the first column to ``Proceedings of the...''
% \usepackage[accepted]{icml2014}

\usepackage{enumerate}
\usepackage{amsmath}
\usepackage{amssymb}
\usepackage{framed}
\usepackage{url}
\usepackage{multirow}
%\usepackage{algpseudocode}

%\COMMENT begins comment
%\ENDCOMMENT ends comment
\long\def\MYCOMMENT#1\ENDMYCOMMENT{\message{(Commented text...)}\par}

\def\R{\mathbb{R}}
\def\S{\mathbb{S}}
\def\xi{\boldsymbol{x}_i}
\def\x{\boldsymbol{x}}
\def\w{\boldsymbol{w}}
\def\p{\boldsymbol{p}}
\def\l{\boldsymbol{l}}
\def\t{\boldsymbol{t}}
\def\X{\boldsymbol{X}}
\def\D{\boldsymbol{D}}

\newcommand{\BB}{BnB}
\newcommand{\sign}{\operatorname{sign}}

\def\argmax{\operatornamewithlimits{arg\,max}}
\def\argmin{\operatornamewithlimits{arg\,min}}

% The \icmltitle you define below is probably too long as a header.
% Therefore, a short form for the running title is supplied here:
\icmltitlerunning{A Probabilistic Model for High Precision and Recall Feature Selection}

\begin{document} 

\twocolumn[
\icmltitle{A Probabilistic Model for High Precision and Recall\\ Feature Selection}

% It is OKAY to include author information, even for blind
% submissions: the style file will automatically remove it for you
% unless you've provided the [accepted] option to the icml2014
% package.
\icmlauthor{Rodridgo Santa Cruz}{rfsantacruz@gmail.com}
\icmladdress{???, Brazil}
\icmlauthor{Scott Sanner}{ssanner@nicta.com.au}
\icmladdress{NICTA \& ANU, Canberra, ACT 2601, Australia}
\icmlauthor{Libo Yin}{21164183@student.uwa.edu.au}
\icmladdress{???, Australia}

% You may provide any keywords that you 
% find helpful for describing your paper; these are used to populate 
% the "keywords" metadata in the PDF but will not be shown in the document
\icmlkeywords{classification, feature selection, graphical models}

\vskip 0.3in
]

\begin{abstract} 
\input abstract
\end{abstract} 

%% Main contents
\input intro
\input background
\input model
\input empirical
\input related_work
\input conclusions

%\section*{Acknowledgments}
%NICTA is funded by the
% Australian Government as represented by
%the Department of Broadband, Communications and the Digital
%Economy and the Australian Research Council through the ICT
%Centre of Excellence program. 

% Bibliography
\bibliography{highprfs}
\bibliographystyle{icml2014}

\end{document} 


